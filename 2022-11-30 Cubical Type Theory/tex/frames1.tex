\begin{frame}{The (De Morgan) Interval}
	\centering
	\uncover<2->{
	We define an \alert{exotype} $\mathbb{I}$, such that:
	
	\cleanalign{\centering}{
		\begin{gathered}
			0, 1 : \mathbb{I}
		\end{gathered}
	}
	}

	\uncover<3->{and equip it with \textbf{connections} \uncover<4->{and a \textbf{reversal}.}}%

	\interrule

	\vspace{\baselineskip}

	\begin{columns}[onlytextwidth]
		\begin{column}{0.5\textwidth}
			\uncover<3->{
			\cleanalign{\centering}{
				\begin{gathered}
					r \lor s : \mathbb{I}\\
					r \land s : \mathbb{I}
				\end{gathered}
			}
			\centering
			
			\vspace{\baselineskip}

			for any $r, s : \mathbb{I}$
			}
		\end{column}
		\begin{column}{0.5\textwidth}
			\uncover<4->{
			\cleanalign{\centering}{
				\begin{gathered}
					\neg t : \mathbb{I}
				\end{gathered}
			}
			\centering
			
			\vspace{\baselineskip}

			for any $t : \mathbb{I}$
			}
		\end{column}
	\end{columns}
\end{frame}

\begin{frame}{The Interval, Morally}
	A \textbf{dimension term} $r : \mathbb{I}$ can be thought of as an point $r \in [0, 1]$.

	\begin{itemize}
		\uncover<2->{
			\item \textbf{Max connection.} $r \lor s $ represents $\mathsf{max}(r, s)$
			\item \textbf{Min connection.} $r \land s$ represents $\mathsf{min}(r, s)$
		}
		\uncover<3->{
			\item \textbf{Reversal.} $\neg r \approx 1 - r$
		}
	\end{itemize}
\end{frame}

\begin{frame}{Points, Lines, Squares, Cubes}
	\centering
	In any context $\Gamma$,
	we have $n$ \textbf{dimension variables} $i : \mathbb{I} \in \Gamma$.
	
	\uncover<2->{
	A term $\Gamma \entails x : A$ looks like:
	}
	\vspace*{\baselineskip}

	\begin{columns}[t,onlytextwidth]
		\uncover<2->{
		\begin{column}{0.333\textwidth}
			\centering
			\begin{tikzpicture}[scale=1,black,x=1pt,y=1pt,every node/.style={scale=0.6}]
				\useasboundingbox (-32, -4) rectangle (80, 80);
				\fill[color_shaded_default]
					(24, 48) circle(2pt)
					;
			\end{tikzpicture}
		\end{column}
		}
		\uncover<3->{
		\begin{column}{0.333\textwidth}
			\centering
			\begin{tikzpicture}[scale=1,black,x=1pt,y=1pt,every node/.style={scale=0.6}]
				\useasboundingbox (-32, -4) rectangle (80, 80);
				% Points
				\coordinate (TL) at (0, 48);
				\coordinate (TR) at (48, 48);
		
				\coordinate (ZI) at ($(TL) + (0, 16)$);
				\coordinate (UI) at ($(TR) + (0, 16)$);
		
				% Square background
				\draw[line width=0,fill=color_fill_default]
				(TL) --
				(TR);
				
				% Interval
				\draw (ZI) -- (UI);
				\draw ($(ZI)+(0,-2)$) -- ($(ZI)+(0,2)$) node[above] {$0$};
				\draw ($(UI)+(0,-2)$) -- ($(UI)+(0,2)$) node[above] {$1$};
				\draw ($(ZI)!0.5!(UI) + (0,2)$) node[above] {$i$}
				;
				
				% Grays
				\draw[line width=1.5pt, color_shaded_default]
					(TL) -- (TR)
					;
			\end{tikzpicture}
		\end{column}
		}
		\uncover<4->{
		\begin{column}{0.333\textwidth}
			\centering
			\begin{tikzpicture}[scale=1,black,x=1pt,y=1pt,every node/.style={scale=0.6}]
				\useasboundingbox (-32, -4) rectangle (80, 80);
				%\draw (-32, -4) rectangle (52, 80);
				% Points
				\coordinate (TL) at (0, 48);
				\coordinate (TR) at (48, 48);
				\coordinate (BR) at (48, 0);
				\coordinate (BL) at (0, 0);
		
				\coordinate (ZI) at ($(TL) + (0, 16)$);
				\coordinate (UI) at ($(TR) + (0, 16)$);
				\coordinate (ZJ) at ($(TL) + (-16, 0)$);
				\coordinate (UJ) at ($(BL) + (-16, 0)$);
				\coordinate (ZJR) at ($(TR) + (16, 0)$);
				\coordinate (UJR) at ($(BR) + (16, 0)$);
		
				% Square background
				\draw[line width=0,fill=color_fill_default]
				(TL) --
				(TR) --
				(BR) --
				(BL) --
				cycle;
				
				% Interval
				\draw (ZI) -- (UI);
				\draw ($(ZI)+(0,-2)$) -- ($(ZI)+(0,2)$) node[above] {$0$};
				\draw ($(UI)+(0,-2)$) -- ($(UI)+(0,2)$) node[above] {$1$};
				\draw ($(ZI)!0.5!(UI) + (0,2)$) node[above] {$i$}
				;
				\draw (ZJ) -- (UJ);
				\draw ($(ZJ)+(2, 0)$) -- ($(ZJ)+(-2, 0)$) node[left] {$0$};
				\draw ($(UJ)+(2,0)$) -- ($(UJ)+(-2, 0)$) node[left] {$1$};
				\draw ($(ZJ)!0.5!(UJ) + (-2, 0)$) node[left] {$j$};
				
				% Grays
				\draw[line width=1.5pt, color_shaded_default]
					(TL) -- (TR) -- (BR) -- (BL) -- cycle
					;
			\end{tikzpicture}
		\end{column}
		}
	\end{columns}
	\vspace*{\baselineskip}
	\uncover<5->{
	A \textbf{line} $i : I \entails x : A$ or $\lambda i.\, x :\mathbb{I} \to A$ \emph{witnesses} the equality of its endpoints.
	}
\end{frame}

\begin{frame}{Cofibrant Propositions}
	\begin{center}
	In any context $\Gamma$,
	we have a class $\mathbb{F}$ of
	
	\textbf{cofibrant propositions}, or \textbf{cofibrations}.
	
	\uncover<2->{We define $\mathbb{F}$ to be the \textbf{face lattice}.}
	\interrule
	\begin{columns}[onlytextwidth]
		\uncover<3->{
		\begin{column}{0.333\textwidth}
			\centering
			\begin{gather*}
				0_\mathbb{F}\\
				1_\mathbb{F}
			\end{gather*}
			\phantom{$\mathbb{I}$}
		\end{column}
		}
		\uncover<4->{
		\begin{column}{0.333\textwidth}
			\centering
			\begin{gather*}
				(i = 0)\\
				(i = 1)
			\end{gather*}
			for any $i : \mathbb{I}$ in $\Gamma$
		\end{column}
		}
		\uncover<5->{
		\begin{column}{0.333\textwidth}
			\centering
			\begin{gather*}
				\varphi \lor \psi\\
				\varphi \land \psi
			\end{gather*}
			for any $\varphi, \psi : \mathbb{F}$
		\end{column}
		}
	\end{columns}
	\end{center}
\end{frame}

\begin{frame}{Partial Elements}
	\centering
	If $\Gamma \entails \varphi : \mathbb{F}$ is a cofibration, \uncover<2->{then $\Gamma, \varphi$ is an \textbf{restricted context}.}
	
	\uncover<3->{We say that $\Gamma, \varphi \entails x : A$ is a \textbf{partial element} of $A$ (on the \alert{extent} of $\varphi$).}
	\vspace*{\baselineskip}
	\begin{columns}[onlytextwidth]
		\uncover<4->{
		\begin{column}{0.5\textwidth}
			\centering
			\begin{tikzpicture}[scale=1,black,x=1pt,y=1pt,every node/.style={scale=0.6}]
				\useasboundingbox (-48, -4) rectangle (80, 80);
				%\draw (-48, -4) rectangle (80, 80);
				% Points
				\coordinate (TL) at (0, 48);
				\coordinate (TR) at (48, 48);
				\coordinate (BR) at (48, 0);
				\coordinate (BL) at (0, 0);
		
				\coordinate (ZI) at ($(TL) + (0, 16)$);
				\coordinate (UI) at ($(TR) + (0, 16)$);
				\coordinate (ZJ) at ($(TL) + (-16, 0)$);
				\coordinate (UJ) at ($(BL) + (-16, 0)$);
				\coordinate (ZJR) at ($(TR) + (16, 0)$);
				\coordinate (UJR) at ($(BR) + (16, 0)$);
		
				% Square background
				\draw[line width=0,fill=color_fill_default]
				(TL) --
				(TR) --
				(BR) --
				(BL) --
				cycle;
				
				% Interval
				\draw (ZI) -- (UI);
				\draw ($(ZI)+(0,-2)$) -- ($(ZI)+(0,2)$) node[above] {$0$};
				\draw ($(UI)+(0,-2)$) -- ($(UI)+(0,2)$) node[above] {$1$};
				\draw ($(ZI)!0.5!(UI) + (0,2)$) node[above] {$i$}
				;
				\draw (ZJ) -- (UJ);
				\draw ($(ZJ)+(2, 0)$) -- ($(ZJ)+(-2, 0)$) node[left] {$0$};
				\draw ($(UJ)+(2,0)$) -- ($(UJ)+(-2, 0)$) node[left] {$1$};
				\draw ($(ZJ)!0.5!(UJ) + (-2, 0)$) node[left] {$j$}
				node[left=16pt,style={scale=1.25}] {$0_{\mathbb{F}}$};
				
				% Grays
				\draw[line width=1.5pt, color_shaded_default]
					(TL) -- (TR) -- (BR) -- (BL) -- cycle
					;
			\end{tikzpicture}
		\end{column}
		}
		\uncover<5->{
		\begin{column}{0.5\textwidth}
			\centering
			\begin{tikzpicture}[scale=1,black,x=1pt,y=1pt,every node/.style={scale=0.6}]
				\useasboundingbox (-48, -4) rectangle (80, 80);
				%\draw (-48, -4) rectangle (80, 80);
				% Points
				\coordinate (TL) at (0, 48);
				\coordinate (TR) at (48, 48);
				\coordinate (BR) at (48, 0);
				\coordinate (BL) at (0, 0);
		
				\coordinate (ZI) at ($(TL) + (0, 16)$);
				\coordinate (UI) at ($(TR) + (0, 16)$);
				\coordinate (ZJ) at ($(TL) + (-16, 0)$);
				\coordinate (UJ) at ($(BL) + (-16, 0)$);
				\coordinate (ZJR) at ($(TR) + (16, 0)$);
				\coordinate (UJR) at ($(BR) + (16, 0)$);
		
				% Square background
				\draw[line width=0,fill=color_heliotrope]
				(TL) --
				(TR) --
				(BR) --
				(BL) --
				cycle;
				
				% Interval
				\draw (ZI) -- (UI);
				\draw ($(ZI)+(0,-2)$) -- ($(ZI)+(0,2)$) node[above] {$0$};
				\draw ($(UI)+(0,-2)$) -- ($(UI)+(0,2)$) node[above] {$1$};
				\draw ($(ZI)!0.5!(UI) + (0,2)$) node[above] {$i$}
				;
				\draw (ZJ) -- (UJ);
				\draw ($(ZJ)+(2, 0)$) -- ($(ZJ)+(-2, 0)$) node[left] {$0$};
				\draw ($(UJ)+(2,0)$) -- ($(UJ)+(-2, 0)$) node[left] {$1$};
				\draw ($(ZJ)!0.5!(UJ) + (-2, 0)$) node[left] {$j$}
				node[left=16pt,style={scale=1.25}] {$1_{\mathbb{F}}$};
				
				% Dashed projections
				\draw[dashed]
					(ZI) -- (TL)
					(UI) -- (TR)
					(ZJ) -- (TL)
					(UJ) -- (BL)
					;
		
				% Highlights
				\draw[line width=1.5pt, color_electric_purple]
					(BL) -- (BR) -- (TR) -- (TL) -- cycle;
				\fill[color_electric_purple]
					(TL) circle(2pt)
					(TR) circle(2pt)
					(BL) circle(2pt)
					(BR) circle(2pt)
					;
				\draw[color_flame, line width=1.5pt]
					(ZI) -- (UI)
					;
				\fill[color_flame]
					(ZI) circle(2pt)
					(UI) circle(2pt)
					;
				\draw[color_cerulean_crayola, line width=1.5pt]
					(ZJ) -- (UJ)
					;
				\fill[color_cerulean_crayola]
					(ZJ) circle(2pt)
					(UJ) circle(2pt)
					;
			\end{tikzpicture}
		\end{column}
		}
	\end{columns}
\end{frame}

\begin{frame}{Example | Cofibrations}
	\centering
	\begin{columns}
		% (i = 0)
		\uncover<2->{
		\begin{column}{0.25\textwidth}
			\centering
			\begin{tikzpicture}[scale=1,black,x=1pt,y=1pt,every node/.style={scale=0.6}]
				\useasboundingbox (-32, -4) rectangle (80, 96);
				% Points
				\coordinate (TL) at (0, 48);
				\coordinate (TR) at (48, 48);
				\coordinate (BR) at (48, 0);
				\coordinate (BL) at (0, 0);
		
				\coordinate (ZI) at ($(TL) + (0, 16)$);
				\coordinate (UI) at ($(TR) + (0, 16)$);
				\coordinate (ZJ) at ($(TL) + (-16, 0)$);
				\coordinate (UJ) at ($(BL) + (-16, 0)$);
				\coordinate (ZJR) at ($(TR) + (16, 0)$);
				\coordinate (UJR) at ($(BR) + (16, 0)$);
		
				% Square background
				\draw[line width=0,fill=color_fill_default]
				(TL) --
				(TR) --
				(BR) --
				(BL) --
				cycle;
				
				% Interval
				\draw (ZI) -- (UI);
				\draw ($(ZI)+(0,-2)$) -- ($(ZI)+(0,2)$) node[above] {$0$};
				\draw ($(UI)+(0,-2)$) -- ($(UI)+(0,2)$) node[above] {$1$};
				\draw ($(ZI)!0.5!(UI) + (0,2)$) node[above] {$i$}
					node[above=16pt,style={scale=1.25}] {$(i = 0)$}
				;
				\draw (ZJ) -- (UJ);
				\draw ($(ZJ)+(2, 0)$) -- ($(ZJ)+(-2, 0)$) node[left] {$0$};
				\draw ($(UJ)+(2,0)$) -- ($(UJ)+(-2, 0)$) node[left] {$1$};
				\draw ($(ZJ)!0.5!(UJ) + (-2, 0)$) node[left] {$j$};
				
				% Dashed projections
				\draw[dashed]
					(ZJ) -- (TL)
					(UJ) -- (BL)
					(ZI) -- (TL)
					;
		
				% Grays
				\draw[line width=1.5pt, color_shaded_default]
					(TL) -- (TR) -- (BR) -- (BL) -- cycle
					;
		
				% Highlights
				\draw[line width=1.5pt, magenta]
					(TL) -- (BL);
				\fill[magenta]
					(TL) circle(2pt)
					(BL) circle(2pt)
					;
				\draw[color_flame, line width=1.5pt]
					(ZJ) -- (UJ)
					;
				\fill[color_flame]
					(ZI) circle(2pt)
					;
					
				\draw[color_cerulean_crayola, line width=1.5pt]
					(ZJ) -- (UJ)
					;
				\fill[color_cerulean_crayola]
					(ZJ) circle(2pt)
					(UJ) circle(2pt)
					;
			\end{tikzpicture}
		\end{column}
		}
		% (j = 1)
		\uncover<3->{
		\begin{column}{0.25\textwidth}
			\centering
			\begin{tikzpicture}[scale=1,black,x=1pt,y=1pt,every node/.style={scale=0.6}]
				\useasboundingbox (-32, -4) rectangle (80, 96);
				% Points
				\coordinate (TL) at (0, 48);
				\coordinate (TR) at (48, 48);
				\coordinate (BR) at (48, 0);
				\coordinate (BL) at (0, 0);
		
				\coordinate (ZI) at ($(TL) + (0, 16)$);
				\coordinate (UI) at ($(TR) + (0, 16)$);
				\coordinate (ZJ) at ($(TL) + (-16, 0)$);
				\coordinate (UJ) at ($(BL) + (-16, 0)$);
				\coordinate (ZJR) at ($(TR) + (16, 0)$);
				\coordinate (UJR) at ($(BR) + (16, 0)$);
		
				% Square background
				\draw[line width=0,fill=color_fill_default]
				(TL) --
				(TR) --
				(BR) --
				(BL) --
				cycle;
				
				% Interval
				\draw (ZI) -- (UI);
				\draw ($(ZI)+(0,-2)$) -- ($(ZI)+(0,2)$) node[above] {$0$};
				\draw ($(UI)+(0,-2)$) -- ($(UI)+(0,2)$) node[above] {$1$};
				\draw ($(ZI)!0.5!(UI) + (0,2)$) node[above] {$i$}
					node[above=16pt,style={scale=1.25}] {$(j = 1)$}
				;
				\draw (ZJ) -- (UJ);
				\draw ($(ZJ)+(2, 0)$) -- ($(ZJ)+(-2, 0)$) node[left] {$0$};
				\draw ($(UJ)+(2,0)$) -- ($(UJ)+(-2, 0)$) node[left] {$1$};
				\draw ($(ZJ)!0.5!(UJ) + (-2, 0)$) node[left] {$j$};
				
				% Dashed projections
				\draw[dashed]
					(ZI) -- (TL)
					(UI) -- (TR)
					(UJ) -- (BL)
					;
		
				% Grays
				\draw[line width=1.5pt, color_shaded_default]
					(TL) -- (TR) -- (BR) -- (BL) -- cycle
					;
		
				% Highlights
				\draw[color_blue, line width=1.5pt]
					(BL) -- (BR);
				\fill[color_blue]
					(BL) circle(2pt)
					(BR) circle(2pt)
					;
				\draw[color_flame, line width=1.5pt]
					(ZI) -- (UI)
					;
				\fill[color_flame]
					(ZI) circle(2pt)
					(UI) circle(2pt)
					;
				\fill[color_cerulean_crayola]
					(UJ) circle(2pt)
					;
			\end{tikzpicture}
		\end{column}
		}
		% (i = 0) OR (j = 1)
		\uncover<4->{
		\begin{column}{0.25\textwidth}
			\centering
			\begin{tikzpicture}[scale=1,black,x=1pt,y=1pt,every node/.style={scale=0.6}]
				\useasboundingbox (-32, -4) rectangle (80, 96);
				% Points
				\coordinate (TL) at (0, 48);
				\coordinate (TR) at (48, 48);
				\coordinate (BR) at (48, 0);
				\coordinate (BL) at (0, 0);
		
				\coordinate (ZI) at ($(TL) + (0, 16)$);
				\coordinate (UI) at ($(TR) + (0, 16)$);
				\coordinate (ZJ) at ($(TL) + (-16, 0)$);
				\coordinate (UJ) at ($(BL) + (-16, 0)$);
				\coordinate (ZJR) at ($(TR) + (16, 0)$);
				\coordinate (UJR) at ($(BR) + (16, 0)$);
		
				% Square background
				\draw[line width=0,fill=color_fill_default]
				(TL) --
				(TR) --
				(BR) --
				(BL) --
				cycle;
				
				% Interval
				\draw (ZI) -- (UI);
				\draw ($(ZI)+(0,-2)$) -- ($(ZI)+(0,2)$) node[above] {$0$};
				\draw ($(UI)+(0,-2)$) -- ($(UI)+(0,2)$) node[above] {$1$};
				\draw ($(ZI)!0.5!(UI) + (0,2)$) node[above] {$i$}
					node[above=16pt,style={scale=1.25}] {$(i = 0) \lor (j = 1)$}
				;
				\draw (ZJ) -- (UJ);
				\draw ($(ZJ)+(2, 0)$) -- ($(ZJ)+(-2, 0)$) node[left] {$0$};
				\draw ($(UJ)+(2,0)$) -- ($(UJ)+(-2, 0)$) node[left] {$1$};
				\draw ($(ZJ)!0.5!(UJ) + (-2, 0)$) node[left] {$j$};
				
				% Dashed projections
				\draw[dashed]
					(ZJ) -- (TL)
					(UJ) -- (BL)
					(ZI) -- (TL)
					(UI) -- (TR)
					;
		
				% Grays
				\draw[line width=1.5pt, color_shaded_default]
					(TL) -- (TR) -- (BR) -- (BL) -- cycle
					;
		
				% Highlights
				\draw[line width=1.5pt, color_flame!33] (ZI) -- (UI);
				\draw[line width=1.5pt, color_cerulean_crayola!33] (ZJ) -- (UJ);
				\fill[color_flame!33]
					(UI) circle(2pt)
					;
				\fill[color_cerulean_crayola!33]
					(ZJ) circle(2pt)
					;

				\draw[line width=1.5pt, magenta]
					(TL) -- (BL);
				\fill[magenta]
					(TL) circle(2pt)
					;
				\draw[line width=1.5pt, color_blue]
					(BL) -- (BR);
				\fill[color_blue]
					(BR) circle(2pt)
					;
				%\begin{scope}
				%	\clip ($(BL) + (4, 4)$) -- ($(BL) + (-4, -4)$) -- ($(BL) + (-4, 4)$) -- cycle;
				%	\fill[magenta] (BL) circle(2pt);
				%\end{scope}
				%\begin{scope}
				%	\clip ($(BL) + (4, 4)$) -- ($(BL) + (-4, -4)$) -- ($(BL) + (4, -4)$) -- cycle;
				%	\fill[color_blue] (BL) circle(2pt);
				%\end{scope}
				\fill[color_electric_purple] (BL) circle(2pt);
				
				\fill[color_flame]
					(ZI) circle(2pt)
					;
				\fill[color_cerulean_crayola]
					(UJ) circle(2pt)
					;
			\end{tikzpicture}
		\end{column}
		}
		% (i = 0) AND (j = 1)
		\uncover<5->{
		\begin{column}{0.25\textwidth}
			\centering
			\begin{tikzpicture}[scale=1,black,x=1pt,y=1pt,every node/.style={scale=0.6}]
				\useasboundingbox (-32, -4) rectangle (80, 96);
				% Points
				\coordinate (TL) at (0, 48);
				\coordinate (TR) at (48, 48);
				\coordinate (BR) at (48, 0);
				\coordinate (BL) at (0, 0);
		
				\coordinate (ZI) at ($(TL) + (0, 16)$);
				\coordinate (UI) at ($(TR) + (0, 16)$);
				\coordinate (ZJ) at ($(TL) + (-16, 0)$);
				\coordinate (UJ) at ($(BL) + (-16, 0)$);
				\coordinate (ZJR) at ($(TR) + (16, 0)$);
				\coordinate (UJR) at ($(BR) + (16, 0)$);
		
				% Square background
				\draw[line width=0,fill=color_fill_default]
				(TL) --
				(TR) --
				(BR) --
				(BL) --
				cycle;
				
				% Interval
				\draw (ZI) -- (UI);
				\draw ($(ZI)+(0,-2)$) -- ($(ZI)+(0,2)$) node[above] {$0$};
				\draw ($(UI)+(0,-2)$) -- ($(UI)+(0,2)$) node[above] {$1$};
				\draw ($(ZI)!0.5!(UI) + (0,2)$) node[above] {$i$}
					node[above=16pt,style={scale=1.25}] {$(i = 0) \land (j = 1)$}
				;
				\draw (ZJ) -- (UJ);
				\draw ($(ZJ)+(2, 0)$) -- ($(ZJ)+(-2, 0)$) node[left] {$0$};
				\draw ($(UJ)+(2,0)$) -- ($(UJ)+(-2, 0)$) node[left] {$1$};
				\draw ($(ZJ)!0.5!(UJ) + (-2, 0)$) node[left] {$j$};
				
				% Dashed projections
				\draw[dashed]
					(ZI) -- (TL)
					(UJ) -- (BL)
					;
		
				% Grays
				\draw[line width=1.5pt, color_shaded_default]
					(TL) -- (TR) -- (BR) -- (BL) -- cycle
					;
		
				% Highlights
				\fill[color_electric_purple]
					(BL) circle(2pt)
					;
				\fill[color_flame]
					(ZI) circle(2pt)
					;
				\fill[color_cerulean_crayola]
					(UJ) circle(2pt);
			\end{tikzpicture}
		\end{column}
		}
	\end{columns}
\end{frame}

% Partial Element notation due to 1802.01170
\begin{frame}{Example | Partial Element}
	\centering
	
	We can define a \textbf{partial element} by pattern matching on a cofibration.

	\interrule

	\uncover<2->{
	For any $i : \mathbb{I} \vdash x, y : A$,
	}
	\uncover<3->{
	\begin{equation*}
		i : \mathbb{I}, \only<4->{\underbrace}{(i = 0) \lor (i = 1)}\only<4->{_\text{sometimes written $\partial i$}} \entails [(i = 0) \mapsto x, (i = 1) \mapsto y] : A \vphantom{[\underbrace{(i = 0) \lor (i = 1)}_\text{sometimes written $\partial i$}]}
	\end{equation*}
	}
\end{frame}

\begin{frame}{Path Types}
	\cleanalign{\centering}{
		\begin{gathered}
		x =_A y
		\end{gathered}
	}

	\vspace{\baselineskip}

	\interrule

	\begin{center}
	\uncover<2->{A path $p : x =_A y$ is a \alert{function} from $\mathbb{I}$ to $A$ \dots}

	\uncover<3->{\dots with \textbf{endpoints} $p\, \textbf{0} \equiv x$ and $p\, \textbf{1} \equiv y$.}
	\end{center}
\end{frame}

\begin{frame}{Example | Reflexivity, Symmetry}
	\centering
	\uncover<2->{Any self-respecting `equality' ought to be \alert{reflexive}, \alert{symmetric}, and \alert{transitive}.}
	\interrule
	\begin{columns}[onlytextwidth]
		\begin{column}{0.5\textwidth}
			\centering
			\begin{align*}
				\uncover<3->{\mathsf{refl} &\defcolon x = x}\\
				\uncover<4->{\mathsf{refl} &\equiv \lambda i.\ \uncover<5->{x}}
			\end{align*}
			\uncover<3->{for any $x : A$}
		\end{column}
		\begin{column}{0.5\textwidth}
			\centering
			\begin{align*}
				\uncover<3->{\mathsf{sym} &\defcolon (x = y) \to (y = x)}\\
				\uncover<6->{\mathsf{sym} &\equiv \lambda p.\ \uncover<7->{\lambda i.\ p \, (\neg i)}}
			\end{align*}
			\uncover<3->{for any $x, y : A$}
		\end{column}
	\end{columns}
	\vspace*{\medskipamount}
\end{frame}

\begin{frame}{Demonstration | Function Extensionality}
	\centering
	Cubical type theories let us \alert{prove} the principle of \textbf{function extensionality}:
	\begin{equation*}
		\mathsf{funext'} \defcolon \prod_{f : A \to B}\,\prod_{g : A \to B}\,\only<2->{\overbrace}{\Biggl(\prod_{x : A}\,f\,x = g\,x\Biggr)}\only<2->{^\text{pointwise equal}} \to {\only<3->{\underbrace}{(f = g)}\only<3->{_\text{equal!}}} \vphantom{\overbrace{\Biggl(\prod_{x : A}\,f\,x = g\,x\Biggr)}^\text{pointwise equal} \to \underbrace{(f = g)}_\text{equal!}}
	\end{equation*}
	\uncover<4->{Let's prove this in \alert{Agda}!}
\end{frame}